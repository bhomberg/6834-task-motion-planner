\documentclass[12pt]{article}

\usepackage{fancyhdr} % Required for custom headers
\usepackage{float}
\usepackage{lastpage} % Required to determine the last page for the footer
\usepackage{extramarks} % Required for headers and footers
\usepackage{graphicx} % Required to insert images
\usepackage{amssymb}
\usepackage{enumerate}

% Margins
\topmargin=-.5in
\evensidemargin=0in
\oddsidemargin=0in
\textwidth=6.5in
\textheight=9in
\headsep=0.25in


\newcommand{\tab}{\hspace*{2em}}

\title{Combined Task and Motion Planner through an Interface Layer}
\author{Alex Gutierrez, Bianca Homberg, and Veronica Lane}


%% ONE SENTENCE PER LINE TO MAKE THIS WORK NICELY IN GIT

\begin{document}

\maketitle

\section{Introduction}

In order to achieve high level goals a robot must combine task and motion planning. 
A robot uses task planning to determine its long term strategy and motion planning to determine the movements movements it  will execute to achieve the task. 
Effectively combining task and motion planning is an open research problem. 
We developed an interface between the motion and task planning. 
Our system uses off-the-shelf task and motion planners and makes no assumptions about their implementation.
 
An alternative approach to developing an interface between the task and motion planner is for the task planner to discretize the geometric state space. 
However, the number of required discretizations to solve the problem is generally very large and results in extremely large or infeasible problems. 
The interface allows the task planner state space to use an abstract state space which ignores geometry. 
The interface layer translates geometric constraints and passes them to the task planner.

\section{Problem Statement}

In robotics, it is often necessary to create a plan for a series of motions of the robot in order to complete some high level plan.
In order to find a task plan which is actually feasible, the task planner must have some knowledge of the geometry of the world.
Without knowledge of the actual spatial configurations of objects in the world, the task planner would create plans where some actions cannot be completed, due to infeasible motion plans from blocking objects or interference between the robot and the environment.

The high level problem this implementation addresses is to create a combined task and motion planner, following that presented in [1].  
This approach allows the motion planner to give information to the task planner and allows us to keep the continuous variables continuous, rather than discretizing.  
This prevents a combinatorial explosion of the state space that the task planner needs to search over.

The main contribution of this implementation is an interface layer.  
This interface layer is designed to be planner-independent: that is, it can work with any motion planner and task planner, so long as the outputs from each are conformed to the interface specified by the API.  
The interface layer interleaves calls between the task planner and the motion planner, using geometric feasibility determined through the motion planner to affect the state given to the task planner to plan from.

Specifically, we start by generating a high level plan and attempting to generate motion plans corresponding to each action.  
When it's not possible to generate a motion plan, the interface layer determines what geometric constraints are causing the problem and relays this back to the task planner in an appropriate format.  
Then, the task planner replans from that set state and the process continues.  
When necessary, backtracking occurs.



\section{High Level Approach}

Specifically, the interface layer starts by generating a high level plan and attempting to generate motion plans corresponding to each action.  
When it's not possible to generate a motion plan, the interface layer determines what geometric constraints are causing the problem and relays this back to the task planner in an appropriate format.  
Then, the task planner replans from that set state and the process continues.  
When necessary, backtracking occurs.

In order to communicate with the task planner and motion planner, the interface layer utilizes ROS services.  
For the task planner, the interface layer instantiates a server with a given PDDL domain file.  
This domain file specifies the different actions that are feasible along with their preconditions and postconditions.  
For individual service calls, the interface layer takes an internal `state' variable, converts it to PDDL format, and sends it to the server to be solved.  
The task planner service responds with a high level plan or a notification that finding a plan failed.  
If finding a plan failed, the interface layer begins backtracking.  
If a high level plan is found, the interface layer begins searching for refinements of each action as a motion plan.

For the motion planner, the interface layer again communicates via a ROS service.  
The motion planner is given an action, a world state, and a pose or set of poses corresponding to sub-parts of the action.  
If it is possible to perform the action using the intermediary poses specified, the motion plan returns the trajectory.  
If not, it returns an error message.

If the motion plan does not succeed, a variety of poses are tried.  
Poses are generated randomly from a pose generator.  
If no poses are successful, the interface layer determines what is blocking the motion plan from succeeding.  
Specifically, it removes objects from the environment until a feasible motion plan is found.  
If still no feasible motion plan is found, backtracking is required.  
If a feasible motion plan is found with some subset of objects removed, then that information is relayed back to the task planner.  
The task planner replans from the step where we failed to find a motion plan.  
At that state, we add in predicates corresponding to the object obstructions.  
Then, the task planner replans from there.  


\subsection{Simple Example Walkthrough}

Consider the environment in FIGURE.  ***** include Fig 2 from the paper

In this environment, we have two objects, blocks 1 and 2, and our goal is to put block 1 into area S.  
However, we cannot pick up block 1 directly since the walls and block 2 surround it.  
Here we walk through the steps the interface layer takes when solving this problem in order to explain the details of the interface layer's construction.

\begin{enumerate}[i.]

\item Call the task planner to create a high level plan.  
In this case, the task planner doesn't yet know that block 2 is blocking block 1, so the plan is simply:

\begin{enumerate}[1.]
\item PICKUP BLOCK1 with GRIPPER

\item PUTDOWN BLOCK1 on SURFACE S
\end{enumerate}

\item Next step is to call the motion planner to try and find a trajectory corresponding to each of these actions.  
In this case, the motion planner fails to find a motion plan for step 1.  

\item The interface layer goes through removing objects to try to find what was causing the error for that action.  
The interface layer discovers that if it removes block 2, then the motion plan can be created just fine, so block 2 must be obstructing the path.

\item We add a predicate corresponding to BLOCK2 OBSTRUCTS BLOCK1 to the planner's state.

\item Now, we try to find a new high level plan.  
Since block 2 is obstructing block 1, we can't pick up block 1 directly, so the task planner realizes it needs to move block 2 first before moving block 1. 
The new plan it creates is:

\begin{enumerate}[1.]
\item PICKUP BLOCK2 	with GRIPPER
\item PUTDOWN BLOCK2 on SURFACE S
\item PICKUP BLOCK1 with GRIPPER
\item PUTDOWN BLOCK1 on SURFACE S
\end{enumerate}

\item Now, we again go to the motion planner and try to find a motion plan corresponding to every action.
In this case, there are no more obstructions, so the process works!

\item Ultimately, the algorithm returns the high level plan in conjunction with a motion plan trajectory for every step of the high level plan.

\end{enumerate}

\section{Technical Details}

-zoom into features, explain how they work on a technical level

-enough understanding to derive + implement algorithm on their own

things to talk about:

\subsection{Representing Continuous Variables Symbolically}

One of the key insights of this paper is that continuous variables can be referenced symbolically in the task planner within the PDDL domain.  
For instance, consider the pose of the robot: this is a continuous, geometric state.  However, the task planner only keeps track of poses symbolically.  
The task planner knows that there is some GPFG\_obj\_1 -- a Gripper Pose For Grasping -- for object 1, and it doesn't really care what that pose is, so long as it exists.  
The task planner reasons over a variety of these symbolic variables, corresponding to poses and trajectories.  
The motion planner and pose generators take these symbolic variables and associate them with corresponding real locations and motions in the world.  
This is the key insight which allows the task planner to reason about continuous variables without discretizing.  

We will follow [1] in using manipulation as an example environment.  For any planning domain with some continuous variables and some discrete variables, it will be possible to transform the continuous variables to discrete symbolic references to continuous positions.  For this example, we talk through the PICKUP action for a simple manipulator; the PUTDOWN action is analagous.


Veronica: probably should fix this to be a nice alignment environment within LaTeX
$PICKUP(obj_1, gripper, pose_1, pose_2, traj)$

$\; precon \; EMPTY(gripper), ROBOTAT(pose_1), $

$\; \; \;\; \; \; ISGPFG(pose_2, obj), ISMP(traj, pose_1, pose_2), $

$\; \; \; \; \; \;\forall obj' \neg OBSTRUCTS(obj', traj, obj_1)$

$\; effect \; IN(obj_1, gripper), \neg EMPTY(gripper), $

$\; \; \;\; \; \; \forall obj', traj' \neg OBSTRUCTS(obj_1, traj', obj'), $

$\; \; \; \; \; \;\forall obj' traj', tloc' \neg PDOBSTRUCTS(obj_1, traj', obj', tloc')$

Here, we have three continuous variables that we represent symbolically: $traj$, for each of the trajectories, and $pose_1$ and $pose_2$, representing start and end locations for the robot's gripper.
The preconditions and effects can also iterate over all of these symbolic variables, allowing us to mark that a picked up object doesn't obstruct any other objects, etc. 
We have various predicates which describe different things about both the discrete variables and the continuous variables.

$EMPTY(gripper)$ -- this specifies whether the gripper is empty and thus available to hold a block.

$ROBOTAT(pose_1)$ -- this specifies where the robot is.  $pose_1$ is a symbolic reference to a continuous variable.

$ISGPFG(pose, obj)$ -- this specifies whether a pose is a Gripper Pose For Grasping (GPFG) an object.

$ISMP(traj, pose_1, pose_2)$ -- this specifies whether a trajectory is a motion plan from one pose to another.

$OBSTRUCTS(obj_1, traj, obj_2)$ -- this specifies whether an object blocks another object from being picked up along a trajectory.

$PDOBSTRUCTS(obj_1, traj, obj_2, tloc)$ -- this specifies whether an object blocks another object from being put down at a particular location along a trajectory.

$IN(obj_1, gripper)$ -- this specifies whether an object is currently grasped by the gripper.



How do we come up with all of these symbolic references?  
We create as many as we need.  
In this case, we need to have a pose where we can pick up each object and put each object down.  
Neither the interface layer nor the task planner know what the poses correspond to in the real world.
The pose generators instantiate specific poses which the motion planner verifies or discounts if it doesn't work.
Similarly, we have one trajectory for each potential pair of poses.  
However, since the trajectory isn't actually used in any of the effects, we actually can ignore it entirely -- it doesn't change the high level plan and the interface layer will still know to have the motion planner create the relevant trajectories. 
So to make the plan more efficient, we actually cut out the $ISMP$ predicate and all references to trajectories.

How do we pass geometric information from the environment back to the task planner?  
We do this via the $OBSTRUCTS$ and $PDOBSTRUCTS$ predicates.  
These tell the task planner that certain objects are blocking motions related to other objects.  
This allows the task planner to have the robot first move the blocking objects so it can get to the other objects.


\subsection{Pose Generators}

veronica can probably write this section

\subsection{Main Interface Layer}

Insert the figure 4 from the paper!

This figure, included with text from [1], explains the flow of control of the interface layer through a variety of situations, including more than the simple case covers.  
The subfigures in this diagram explain on a high level the different steps the interface layer takes when solving a problem.  
The subfigure a shows the motion planner's backtracking search when trying to instantiate motions for each of the high level actions.  
Subfigure b shows the first possible scenario, where the backtracking search succeeds.  Subfigure c shows a second possible scenario, where the backtracking search doesn't succeed.  
There are multiple options of what can happen within this scenario: subfigure d shows an option where they determine that the block2 obstructs the trajectory.  
This triggers the behavior discussed earlier, where the state is updated for the task planner.  
Subfigure e shows the new task plan after replanning occurs.  
After this, the interface layer will repeat the process of attempting to refine this new plan.  
Subfigure f shows an alternative option, where the blocking object is determined to be the table, which is not movable.  
Then, the state is updated as before and the task planner is called, but no solution is found, so backtracking occurs.

\subsubsection{Algorithm 1}

Include algorithm block 1 from paper

The outer loop of the algorithm interfaces with the high level task planner.  
We start with an initial state where no objects obstruct any motions.  
From there, we look for a high level plan.  
If the problem is not solvable without any problematic environmental information, then the problem is unsolvable and we immediately return that information.  
If we do succeed at finding a high level task plan, we begin the process of attempting to find a motion corresponding to each action.

First, we start by attempting to find a full refinement of the whole plan,calling the TryRefine subroutine in error-free mode.  
If that succeeds, then we return the plan.  

If that doesn't succeed, we find a partial trajectory that goes as far as it can by calling the TryRefine subroutine in partial trajectory mode.  
This function returns a partial trajectory along with the step where it failed and the determined cause for failure.  
Then, we update the state based on the actions up until that point and add in the obstructing obstacles corresponding to the failure.  
Now, we attempt to find a new high level plan, and iterate.  

If we fail to find a new high level plan at any point, we re-run the TryRefine subroutine on the original high level plan from the previous step.  
Since the TryRefine subroutine has randomization from pose generators, re-running it may yield a better solution. 

We have several checks where, if we fail to find a solution, we clear all of our information and start over.  Due to the randomization of the pose generators, we may make better choices during subsequent iterations.  

The original version of the algorithm, as shown in FIGURE, only attempts to find partial trajectories when it replans partway through the procedure.  Our algorithm has a slight modification where, every time a new partial plan is generated, we first try to fill it completely using backtracking search before giving up and resorting to the partial trajectory finding.  In some scenarios, some poses won't work but others will, so it is in our benefit to try and make the simplest possible plan work before giving up and making the plan more complicated.  In our manipulation example, this turns into trying to pick up blocks from all angles before giving up and assuming they must be blocked by something.  An analysis of the benefits of this approach are outlined in the benchmarking section.  

\subsubsection{Algorithm 2}

include algorithm block 2 from paper!  go with the one from the short version of the paper, since it's the version we're actually going with.  the algorithm block is shorter too

In the TryRefine subroutine, the interface layer attempts to find motion plans corresponding to the various actions.  
It has two modes: error free and partial trajectory.  
In error free mode, it performs backtracking search to try to find a full plan from start to finish.  
In partial trajectory mode, it goes as far as it can, but when it finds an infeasible motion plan, it returns the step at which it failed.  
In addition to returning the failed step, it returns the cause of the errors.  
The function MPErrs removes various objects from the environment until it find a plan which succeeds.  
It then marks all of those objects which were removed as blocking objects.

NOTE TO BIANCA: include paragraph about how we utilize pose generators here

there will be another couple paragraphs here extending this


\subsection{API and Interfaces}

Alex should write this section, probably mostly the same as the stuff he's writing for the readme/etc.

\section{Benchmarking}

\subsection{Testing}

\subsection{Procedure}

\subsection{Results}

to include:

-tables of results

-description of how we did benchmarking

-in particular, description of why we did number of calls to task and motion planner instead of total time taken

-discussion of testing procedure as well

\section{Limitations}

-realistic expectations for someone interested in applying the reasoning technique

-still need to get a motion planner and a task planner working and need to make them have the right API interface -- not a pure out of the box working thing

-is not guaranteed to find an optimal solution

-cannot do online re-planning

-time to run is based on speed of the motion planner and task planner

\section{Open Questions}

-include more grasp primitives (e.g. push grasps)

-partially observable or non-deterministic environments

-replanning online

\section{References}

[1] S. Srivastava, E. Fang, L. Riano, R. Chitnis, S. Russell, P. Abbeel. "Combined Task and Motion Planning Through an Extensible Planner-Independent Interface Layer". IEEE International Conference on Robotics and Automation (ICRA), 2014.






\end{document}

